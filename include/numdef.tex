% Save file as: NUMDEF.STY             Source: FILESERV@SHSU.BITNET  
%                    numdef.sty
                     %%%%%%%%%%

% Prefixing a macro definition with \num allows numbers at the end of
% the command name ( At first glance )

% eg:
% \num\def\x1{x one}
% \num\def\x2{x two}
% \num\newcommand{\y100}[1]{y one hundred, with argument #1}

%                   David Carlisle
%                   carlisle@uk.ac.man.cs

% NOTE: this is a LaTeX style option, but may be used with plain TeX.
%   \catcode`\@=11 \input numdef.sty \catcode`\@=12

% SEE NOTES AT END OF FILE
%%%%%%%%%%%%%%%%%%%%%%%%%%%%%%%%%%%%%%%%%%%%%%%%%%%%%%%%%%%%%%%%%%%%%%%%%%%

\def\num@def#1{\def\num@name{#1}\edef\num@str{\string #1}%
  \expandafter\ifx\csname\num@str{*}\endcsname
                  \relax\num@start\fi\futurelet\next\num@d@f}%

\def\num@d@f{\ifx\next*\let\@tempa\num@defA
                \else\ifx\next\bgroup\let\@tempa\num@defB
                \else\let\@tempa\num@defC \fi\fi\@tempa}%

\newcount\num@count
\def\N{\number\num@count}%

\def\num@defA*{\def\num@{*}\num@def@}%
\def\num@defB#1{\let\num@=\N\num@count= #1 \num@def@}%
\def\num@defC{\let\num@=\N\afterassignment\num@def@\num@count= }%

\def\num@def@{\expandafter\num@define\csname\num@str{\num@}\endcsname}%

%%
\def\num#1#2{\let\num@define=#1\num@def#2}%
%%

\def\num@start{%
  \expandafter\ifx\num@name\@undefined\let\next\num@st@rt
  \else\def\next{\errmessage{numdef: \num@str\space is already defined}}%
  \fi\next}%

\def\num@st@rt{\expandafter\def\csname\num@str{*}\endcsname
    {\errmessage{numdef: \num@str\N\space is not defined}}%
 \expandafter\num@use\num@name}%

\def\num@use#1{\def#1{\edef\num@str{\string #1}\futurelet\@tempa\num@us@}}
\def\num@us@{%
  \ifx\@tempa\bgroup\let\next\num@useA\else\let\next\num@useB\fi\next}

\def\num@useA#1{\num@count= #1 \num@use@}%
\def\num@useB{\afterassignment\num@use@\num@count= }%
\def\num@use@{\expandafter\ifx\csname\num@str{\N}\endcsname\relax
  \def\num@{*}\else\let\num@=\N\fi\csname\num@str{\num@}\endcsname}%

%%%%%%%%%%%%%%%%%%%%%%%%%%%%%%%%%%%%%%%%%%%%%%%%%%%%%%%%%%%%%%%%%%%%%%%%%%%
\endinput

\num takes a liberal view of where to put { } pairs. In particular
the <number> and/or the command name may be surrounded by { }.

The token after \num must be a command for defining macros. If a \long
(or \outer) def is required, first define, eg, \def\ldef{\long\def} as
\num\long\def ... does NOT work.
All of the following definitions are equivalent:
\num\ldef\x10#1{x ten (#1)}              \num\ldef\x{10}#1{x ten (#1)}
\num\ldef\x{"A}#1{x ten (#1)}            \num\ldef\x'12#1{x ten (#1)}
\num\newcommand{\x10}[1]{x ten (#1)} \num\newcommand{\x{10}}[1]{x ten (#1)}

To use the macro defined via \num just type the name followed by the
<number>, optionally enclosed by { }.
eg   \x10{abc}   \x{10}{abc}  \X"A{abc}   all produce "x ten (abc)".
If the <number> is not surrounded by { }, the command name gobbles spaces

If there has not been a definition corresponding to the <number>, then a
default command is run, initially this produces an error message, but it
can be redefined using `*' instead of a <number> thus after one of the
above definitions \x2 would generate an error "\x2 is not defined"
If a default definition for \x<number> is required, use a definition
such as
\num\def\x*{the replacement text for x\N}
Here \N stands for the value of the <number>, so now \x2 expands to
"the replacement text for x2"
eg:
\num\def\poly1{degenerate case (n=1)} \num\def\poly2{degenerate case (n=2)}
\num\def\poly3{triangle}              \num\def\poly4{quadrilateral}
\num\def\poly*{$\N$-gon}

Note that this example could more easily be achieved by
\def\poly#1{\ifcase #1\or degenerate case (n=1) \or ...\else $#1$-gon\fi}

\num is only really useful if one needs to have an `array' of macros and
need to update the definition of one macro in the array without altering
the expansions of the others.

Warning, if you use the * form to define a macro with arguments, make sure
that any macro that might change the value of \N is kept local to a { } group
eg
\num\def\z1{NOW}   \num\def\y*#1{N is \N, #1\  N is \N}

\y3{\z1}  % produces "N is 3 NOW N is 1"  \z1 has made \N be 1
\y3{{\z1}}% produces "N is 3 NOW N is 3"  changes to \N are local to a group.
