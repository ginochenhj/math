\begin{definition}\bref{[Noble,AlgLinApl-pt,1986]}
Chama-se de matriz transposta hermitiana, denotado por $A^H$, $A^H = \overline{A}^t$, que é a complexa cojugada da transposta ordinária.
\end{definition}

\begin{definition}\bref{[Noble,AlgLinApl-pt,1986]}
Chama-se de matriz hermitiana uma matriz tal que $A^H = A$.
\end{definition}

\begin{lemma}\bref{[Noble,AlgLinApl-pt,1986]}
Uma matriz hermitiana é a soma de uma matriz real simétrica e de uma matriz imaginária anti-simétrica.
\end{lemma}

\begin{lemma}\bref{[Noble,AlgLinApl-pt,1986]}
As propriedades de matrizes são válidas:
	\begin{itemize}
		\item $\overline{AB} = \overline{A}.\overline{B}$
		\item $(\overline{A})^t = (\overline{A^t})$
	\end{itemize}
\end{lemma}

\begin{definition}\bref{[Noble,AlgLinApl-pt,1986]}
	Uma matriz $P$ tal que $P^HP = PP^H = I$ é chamada matriz unitária. 
\end{definition}

\begin{definition}\bref{[Noble,AlgLinApl-pt,1986]}
	Uma matriz $P$ real tal que $P^TP = PP^T = I$ é chamada matriz ortogonal.
\end{definition}

\begin{theorem}\bref{[Noble,AlgLinApl-pt,1986]}
	\begin{enumerate}[label=(\roman*)]
		\item Tanto as colunas quanto as linhas de uma matriz unitária (ou ortogonal) formal um cojunto ortonormal.
		\item Se $P$ é unitária, então $|det P| = 1$.
		\item Se $P$ e $Q$ são unitárias, então o mesmo acontece com $PQ$.
		\item Se $P$ é unitária, então, para todos os $x$ e $y$, temos $(Px, Py), \lVert Px \rVert_2 = \lVert x \rVert_2$ e $\lVert P \rVert_2 = 1$.
		\item Se $\lambda$ for um autovalor da matriz unitária $P$, então $|\lambda| = 1$.
	\end{enumerate}
\end{theorem}

\begin{definition}\bref{[Noble,AlgLinApl-pt,1986]}
	Uma matriz quadrada $A$ que satisfaz $A^HA = AA^H$ é chamada matriz normal.
\end{definition}

\begin{lemma}\bref{[Noble,AlgLinApl-pt,1986]}
	Seja $A$ uma matriz quadrada.
	\begin{enumerate}[label=(\roman*)]
		\item Matrizes hermitianas são normais, ou seja $A^HA = AA^H = AA$
		\item Se $A$ é uma matriz real, $A^TA = AA^T = AA$, $A$ é simétrico.
	\end{enumerate}
\end{lemma}
