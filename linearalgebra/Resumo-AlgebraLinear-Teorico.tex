\documentclass[10pt,a4paper]{article}
\usepackage[utf8]{inputenc}
\usepackage{amsmath}
\usepackage{amsthm}
\usepackage{amssymb}
\usepackage{amsfonts}
\usepackage{amssymb}
\usepackage{mathrsfs}
\usepackage{mathtools}
\usepackage{graphicx}
\usepackage{geometry}
\usepackage{framed}
\usepackage{enumitem}
\usepackage{tabto}
\usepackage{xcolor}
\usepackage{listings}

\geometry{
	a4paper,
	total={210mm,297mm},
	left=20mm,
	right=20mm,
	top=15mm,
	bottom=15mm,
}

\lstset{basicstyle=\ttfamily,
	 rulesepcolor=\color{black},
	 showspaces=false,showtabs=false,tabsize=2,
	 numberstyle=\tiny,
	 stringstyle=\color{sh_string},
	 keywordstyle = \color{sh_keyword}\bfseries,
	 commentstyle=\color{sh_comment}\itshape,
	 captionpos=b,
	 xleftmargin=0.7cm, xrightmargin=0.5cm,
	 lineskip=-0.3em,
	 escapebegin={\lstsmallmath}, escapeend={\lstsmallmathend}
}




\newtheorem{theorem}{\color{red}{Teorema}}[section]
\newtheorem{lemma}[theorem]{\color{purple!70}{Lema}}
\newtheorem{proposition}[theorem]{\color{purple!70}{Proposição}}
\newtheorem{conjecture}[theorem]{\color{purple!70}{Conjectura}}
\newtheorem{corollary}[theorem]{\color{purple!70}{Corolário}}
\newtheorem{definition}[theorem]{\color{blue!90}{Definição}}

\newtheorem{remark}[theorem]{\color[rgb]{0,.5,0}{Observação}}
\newtheorem{attention}[theorem]{\color[rgb]{.75,0,0}{Atenção}}

\renewcommand*{\proofname}{Demonstração}

\newcommand{\bref}[1]{\textsuperscript{\tiny\textnormal{\textbf{\setlength{\fboxsep}{1pt}\colorbox{gray!50}{\color{white}{#1}}}}}}

%\endinput


\setlist[description]{leftmargin=\parindent,labelindent=\parindent,topsep=0pt,itemsep=-1ex,partopsep=1ex,parsep=1ex}
\setlist[itemize]{leftmargin=\parindent,labelindent=\parindent,topsep=0pt,itemsep=-1ex,partopsep=1ex,parsep=1ex}

%\setlength\parindent{0pt}

\author{Gino Chen Hsiang-Jan}
\title{Resumo Álgebra Linear}
\date{26 de Outubro de 2015}

\begin{document}

%******************************************************************************
% Corpos
%******************************************************************************
\section{Corpos}
\begin{definition}\bref{[Coelho,AlgLin,2013]}
	Um conjunto não vazio $\mathsf{K}$ é um corpo se em $\mathsf{K}$ pudermos definir duas operações, denotadas por +(adição) e . (multiplicação). satisfazendo as seguintes propriedades:
	\begin{description}
		\item[propriedade comutativa] (A1) $a + b = b + a, \forall a, b \in \mathsf{K}$ 
		\item[propriedade associativa] (A2) $a + (b + c) = (a + b) + c, \forall a, b, c \in \mathsf{K}$ 
		\item[elemento neutro da soma] (A3) Existe um elemento em $\mathsf{K}$, denotado por $0$ e chamado de elemento neutro da adição, que satisfaz $0 + a = a + 0 = a, \forall a \in \mathsf{K}$
		\item[inserso aditivo] (A4) Para cada $a \in \mathsf{K}$, existe um número em $\textsf{K}$ , denotado por $-a$ e chamado de oposto de $a$ (ou inverso aditivo de $a$) tal que $a + (-a) = 0$
		\item[propriedade comutativa] (M1) $a.b = b.a, \forall a, b \in \mathsf{K}$ (propriedade comutativa)
		\item[propriedade associativa] (M2) $a.(b.c) = (a.b).c, \forall a, b, c \in \mathsf{K}$ (propriedade associativa)
		\item[elemento neutro da multiplicação] (M3) Existe um elemento em $\mathsf{K}$, denotado por $1$ e chamado de elemento neutro da multiplicação, tal que $1.a = a.1 = a, \forall a \in \mathsf{K}$
		\item[inverso multiplicativo] (M4) Para cada elemento não nulo $a \in \mathsf{K}$, existe um elemento em $\mathsf{K}$, denotado por $a^{-1}$ e chamando de inverso multiplicativo  de $a$, tal que $a.a^{-1} = a^{-1}.a = 1$.
	\end{description}
\end{definition}

%******************************************************************************
% Matrizes
%******************************************************************************
\section{Matrizes}

\begin{definition}\bref{[Noble,AlgLinApl-pt,1986]}
Uma matriz anti-simétrica é definida como sendo uma matriz tal que $A^t = -A$, ou seja, $a_{ij} = -a_{ji}$
\end{definition}

\begin{lemma}\bref{[Noble,AlgLinApl-pt,1986]}
Uma matriz anti-simétrica é quadrada, e que os elementos da diagonal de uma matriz anti-simétrica são nulos.
\end{lemma}

\begin{lemma}\bref{[Noble,AlgLinApl-pt,1986]}
Se $A$ é qualquer matriz quadrada, então $A-A^t$ é anti-simétrica.
\end{lemma}

\begin{definition}\bref{[Noble,AlgLinApl-pt,1986]}
Uma matriz real é uma cujos elementos são todos reais. Uma matriz complexa tem elementos que podem ser complexos. Uma matriz imaginária tem elementos que são todos imaginários ou nulos. O símbolo $\overline{A}$ é usado para represenar a matriz cujo $(i, j)$-ésimo elementos é conjugado complexo $\overline{a}_{ij}$ do $(i, j)$-ésimo elemento de $A$. 
\end{definition}

\begin{definition}\bref{[Noble,AlgLinApl-pt,1986]}
Chama-se de matriz transposta hermitiana, denotado por $A^H$, $A^H = \overline{A}^t$, que é a complexa cojugada da transposta ordinária.
\end{definition}

\begin{definition}\bref{[Noble,AlgLinApl-pt,1986]}
Chama-se de matriz hermitiana uma matriz tal que $A^H = A$.
\end{definition}

\begin{lemma}\bref{[Noble,AlgLinApl-pt,1986]}
Uma matriz hermitiana é a soma de uma matriz real simétrica e de uma matriz imaginária anti-simétrica.
\end{lemma}

\begin{lemma}\bref{[Noble,AlgLinApl-pt,1986]}
As propriedades de matrizes são válidas:
	\begin{itemize}
		\item $\overline{AB} = \overline{A}.\overline{B}$
		\item $(\overline{A})^t = (\overline{A^t})$
	\end{itemize}
\end{lemma}

\begin{definition}\bref{[Noble,AlgLinApl-pt,1986]}
	Uma matriz $P$ tal que $P^HP = PP^H = I$ é chamada matriz unitária. 
\end{definition}

\begin{definition}\bref{[Noble,AlgLinApl-pt,1986]}
	Uma matriz $P$ real tal que $P^TP = PP^T = I$ é chamada matriz ortogonal.
\end{definition}

\begin{theorem}\bref{[Noble,AlgLinApl-pt,1986]}
	\begin{enumerate}[label=(\roman*)]
		\item Tanto as colunas quanto as linhas de uma matriz unitária (ou ortogonal) formal um cojunto ortonormal.
		\item Se $P$ é unitária, então $|det P| = 1$.
		\item Se $P$ e $Q$ são unitárias, então o mesmo acontece com $PQ$.
		\item Se $P$ é unitária, então, para todos os $x$ e $y$, temos $(Px, Py), \lVert Px \rVert_2 = \lVert x \rVert_2$ e $\lVert P \rVert_2 = 1$.
		\item Se $\lambda$ for um autovalor da matriz unitária $P$, então $|\lambda| = 1$.
	\end{enumerate}
\end{theorem}

\begin{definition}\bref{[Noble,AlgLinApl-pt,1986]}
	Uma matriz quadrada $A$ que satisfaz $A^HA = AA^H$ é chamada matriz normal.
\end{definition}

\begin{lemma}\bref{[Noble,AlgLinApl-pt,1986]}
	Seja $A$ uma matriz quadrada.
	\begin{enumerate}[label=(\roman*)]
		\item Matrizes hermitianas são normais, ou seja $A^HA = AA^H = AA$
		\item Se $A$ é uma matriz real, $A^TA = AA^T = AA$, $A$ é simétrico.
	\end{enumerate}
\end{lemma}


%******************************************************************************
% Espaços e Subespaços Vetoriais
%******************************************************************************
\section{Espaços e Subespaços Vetoriais}

\begin{definition}\bref{[Lima,AlgLin,2014]}
	Chama-se espaço vetorial $\textsf{E}$ a um conjunto de elementos chamados vetores, no qual estão
	definidas duas operações: adição, $\forall u, v \in \textsf{E} \Rightarrow u+v \in \textsf{E}$,  e multiplicação $\forall \alpha \in \mathbb{R},  \forall v \in \textsf{E} \Rightarrow \alpha v \in \textsf{E}$. Sejam, $\alpha, \beta \in \mathbb{R}, u, v \in \textsf{E}$. As operações de adição e multiplicação também devem satisfazer os seguintes axiomas:
	\begin{description}
		\item[comutatividade     ] $u+v = v+u$
		\item[associatividade    ] $(u+v)+w = v+(u+w)$ e $(\alpha \beta)v = \alpha (\beta v)$
		\item[vetor nulo         ] existe um vetor $0 \in \textsf{E}$, tal que $v + 0 = 0 + v$ para todo $v \in \textsf{E}$
		\item[inverso aditivo    ] para cada vetor $v \in \textsf{E}$ existe $-v \in \textsf{E}$ tal que $v + (-v) = 0$
		\item[distributividade   ] $(\alpha + \beta)v = \alpha v + \beta v$ e $\alpha(u + v) = \alpha u + \alpha v$
		\item[multiplicação por 1] $1 v = v$
	\end{description}
\end{definition}


\begin{definition}\bref{[Lima,AlgLin,2014]}
	Seja $\textsf{E}$ um espaço vetorial, Um subespaço vetorial de $\textsf{E}$ é um subconjunto $\textsf{F} \subset \textsf{E}$ com as seguintes propriedades:
	\begin{description}
		\item[vetor nulo         ] existe um vetor $0 \in \textsf{F}$
		\item[fechamento da soma] $\forall u, v \in \textsf{F} \Rightarrow u+v \in \textsf{F}$
		\item[fechamento produto com escalar] $\forall v \in \textsf{E} \Rightarrow \forall \alpha \in \mathbb{R}, \alpha v \in \textsf{F}$
	\end{description}
\end{definition}

\begin{definition}\bref{[Lima,AlgLin,2014]}
	Seja $\textsf{F}_1, \textsf{F}_2 \subset \textsf{E}$, $F_1 + F_2 := \{ v_1 + v_2 : v_1 \in F_1 \land v_2 \in F_2 \}$.
\end{definition}
\par
\begin{definition}\bref{[Lima,AlgLin,2014]}
	Seja $\textsf{F}_1, \textsf{F}_2 \subset \textsf{E}$ $\land$ $\textsf{F}_1 \cap \textsf{F}_2 = \{0_v\}$, chama-se de soma direta entre $\textsf{F}_1$ e $\textsf{F}_2$, denotado como $\textsf{F}_1 \oplus \textsf{F}_2$, $\textsf{F}_1 \oplus \textsf{F}_2 := \textsf{F}_1 \cup \textsf{F}_2$.
\end{definition}

\begin{theorem}\bref{[Lima,AlgLin,2014]}
	Sejam $\textsf{F}, \textsf{F}_1, \textsf{F}_2$ subespaços vetoriais de $\textsf{E}$ com $\textsf{F}_1 \subset \textsf{F}$ e $\textsf{F}_2 \subset \textsf{F}$. São equivalentes:
	\begin{itemize}
		\item $\textsf{F} = \textsf{F}_1 \oplus \textsf{F}_2$
		\item $\forall w \in \textsf{F} $ se escreve, de modo único, como a soma $w = v_1 + v_2$, onde $v_1 \in \textsf{F}_1$, e $v_2 \in \textsf{F}_2$
	\end{itemize}
\end{theorem}	

\begin{definition}\bref{[Lima,AlgLin,2014]}
	Seja $\textsf{E}$ um espaço vetorial. Se $x, y \in \textsf{E}$ e $x \neq y$, a reta que une os pontos $x, y$ é, por definição o conjunto: $r = \{(1 - t)x + ty; t \in \mathbb{R}\}$.
\end{definition}

\begin{definition}\bref{[Lima,AlgLin,2014]}
	Um subconjunto $\textsf{V} \in \textsf{E}$ chama-se uma variedade afim quando a reta que une dois pontos quaisquer de $\textsf{V}$ está contida em $\textsf{V}$. Assim $\textsf{V} \in \textsf{E}$ é uma variedade afim se, e somente se, cumpre a seguinte condição:
	\[
		x, y \in \textsf{V}, t \in \mathbb{R} \Rightarrow (1 - t)x + ty \in \textsf{V}.
	\]
\end{definition}

\begin{theorem}\bref{[Lima,AlgLin,2014]}
	Se $\textsf{V}_1, \textsf{V}_2, \textsf{V}_3, \dots, \textsf{V}_n  \subset \textsf{E}$ são variedades afins, estão a interseção $V = \textsf{V}_1 \cap \textsf{V}_2 \cap \textsf{V}_3 \cap \dots, \cap \textsf{V}_n$ também é uma variedade afim.
\end{theorem}

\begin{theorem}\bref{[Lima,AlgLin,2014]}
	Todo ponto $p \in \textsf{E}$ é uma variedade afim.
\end{theorem}

\begin{theorem}\bref{[Lima,AlgLin,2014]}
	Seja $\textsf{V}$ uma variedade afim não-vazia no espaço vetorial $\textsf{V}$. Existe um único subespaço vetorial $\textsf{F} \subset \textsf{E}$ tal que, $\forall x \in \textsf{V}$ tem-se $\textsf{V} = x + \textsf{F} = \{x + v: v\in \textsf{F} \}$.
\end{theorem}


\section{Bases}
Considere $V$ um espaço vertorial sobre $\mathsf{K}$.

\begin{definition}\bref{[Coelho,AlgLin,2013]}
	Um vetor $v \in \mathsf{V}$ é uma combinação linear dos vetores $v_1, \dots, v_n \in \mathsf{V}$ se existirem escalares $\alpha_1, \dots \alpha_n \in \mathsf{K}$ tais que:
	\[
		v = \alpha_1 v_1 + \dots + \alpha_n v_n = \sum_{i = 0}^{n} \alpha_i v_i
	\]
\end{definition}

\begin{definition}\bref{[Coelho,AlgLin,2013]}
		Seja $\mathcal{B}$ um subconjunto de $\mathsf{V}$. Dizemos que $\mathcal{B}$ é um conjunto gerador de $\mathsf{V}$ se todos os elementos de $\mathsf{V}$ for uma combinação linear de um número finito de elementos de $\mathcal{B}$.
\end{definition}

\begin{definition}\bref{[Coelho,AlgLin,2013]}
	O conjunto vazio gera o espaço vetorial ${0}$.
\end{definition}

\begin{definition}\bref{[Coelho,AlgLin,2013]}
	Seja $\mathcal{B}$ um subconjunto de $\mathsf{V}$. Dizemos que $\mathcal{B}$ é linearmente independente (ou l.i.) se $\alpha_1 v_1 + \dots + \alpha_n v_n = 0$, para $v_i \in \mathcal{B}$ e $\alpha_i \in \mathsf{K}$, $i = 0, \dots, n$ implica $\alpha_1 = \dots = \alpha_n = 0$.
\end{definition}

\begin{definition}\bref{[Lima,AlgLin,2014]}
	Seja $\textsf{E}$ um espaço vetorial. Diz-se que um conjunto $X \subset \textsf{E}$ é linearmente independente (LI) quando nenhum vetor $v \in X$ é combinação linear de outros elementos de $X$ ou se $X$ tem somente um elemento não nulo.\\
	
	\bref{[Coelho,AlgLin,2013]} Dizemos que um subconjunto $\mathcal{B}$ de $\mathsf{V}$ é uma base de $\mathsf{V}$ se:\\
	(i) $\mathcal{B}$ for um subconjunto gerador de $\mathsf{V}$\\
	(ii) $\mathcal{B}$ for linearmente independente.
\end{definition}

\begin{definition}\bref{[Coelho,AlgLin,2013]}
	O conjunto vazio é uma base do espaço vetorial ${0}$.
\end{definition}

\begin{theorem}\bref{[Lima,AlgLin,2014]}
	Seja $X$ um conjunto LI de um espaço vetorial $\textsf{E}$. $\alpha_1 v_1 + \alpha_2 v_2 + \alpha_3 v_3 + \dots + \alpha_m v_m$ com $v_1, v_2, v_3, \dots, v_m \in X \Leftrightarrow \alpha_1 = \alpha_2 = \alpha_3 = \dots = \alpha_m = 0$
\end{theorem}

\begin{corollary}\bref{[Lima,AlgLin,2014]}
	Se $v = \alpha_1 v_1 + \alpha_2 v_2 + \alpha_3 v_3 + \dots + \alpha_m v_m = \beta_1 v_1 + \beta_2 v_2 + \beta_3 v_3 + \dots + \beta_m v_m$ e os vetores $v_1, v_2, v_3, \dots, v_m $ são LI então $\alpha_1 = \beta_1, \alpha_2 = \beta_2, \alpha_3 = \beta_3, \dots, \alpha_m = \beta_m$.
\end{corollary}

\begin{theorem}\bref{[Lima,AlgLin,2014]}
	Sejam $v_1, v_2, v_3, \dots, v_m $ vetores não nulos do espaço vetorial $\textsf{E}$. Se nenhum deles é combinação linear dos anteriores então $X = \{v_1, v_2, v_3, \dots, v_m \}$ é LI.
\end{theorem}

\begin{definition}\bref{[Lima,AlgLin,2014]}
	Um conjunto $X \in \textsf{E}$ é linearmente dependente (LD) quando não é LI. Ou seja, $X = \{0_v\}$ ou $\exists v \in X$ tal que $v$ é combinação linear de outros elementos em $X$.
\end{definition}

\begin{definition}\bref{[Lima,AlgLin,2014]}
	Uma base de um espaço vetorial $\textsf{E}$ é um conjunto $\mathcal{B} \subset \textsf{E}$ linearmente independente que gera $\textsf{E}$. Ou seja, $\forall v \in \textsf{E}$ se exprime, de modo único como combinação linear $v = \alpha_1 v_1 + \alpha_2 v_2 + \alpha_3 v_3 + \dots + \alpha_m v_m$, $v_1, v_2, v_3,\dots, v_m \in \mathcal{B}$. $\alpha_1, \alpha_2, \alpha_3, \dots, \alpha_m$ são coordenadas do vetor $v$ na base $\mathcal{B}$.
\end{definition}

\begin{definition}\bref{[Coelho,AlgLin,2013]}
	Dizemos que $\mathsf{V}$ é finitamente gerado se possuir um gerador finito.
\end{definition}

\begin{proposition}\bref{[Coelho,AlgLin,2013]}
	Seja $\mathsf{V}$ um $\mathsf{K}$ espaço vetorial finitamente gerado não nulo e assuma $\{v_1, \dots, v_m\}$ seja um conjunto gerador de $\mathsf{V}$. Então todo conjunto linearmente independente de vetores em $\mathsf{V}$ tem no máximo $m$ elementos.
\end{proposition}

\begin{corollary}\bref{[Coelho,AlgLin,2013]}
	Seja $\mathsf{V}$ um $\mathsf{K}$ espaço vetorial finitamente gerado não nulo. Então duas bases qualquer de $\mathsf{V}$ têm o mesmo número de elementos.
\end{corollary}

\begin{definition}\bref{[Lima,AlgLin,2014]}
	Um sistema linear é chamado homogêneo quando o segundo membro de cada equação é igual a zero.
\end{definition}

\begin{theorem}\bref{[Lima,AlgLin,2014]}
	Todo sistema linear homogêneo cujo número de incógnitas é maior do que o número de equações admite uma solução não trivial.
\end{theorem}

\begin{theorem}\bref{[Lima,AlgLin,2014]}
	Se os vetores $v_1, v_2, v_3, \dots, v_m$ geram o espaço vetorial $\textsf{E}$ então qualquer conjunto com mais de $m$ vetores em $\textsf{E}$ é LD
\end{theorem}

\begin{theorem}\bref{[Lima,AlgLin,2014]}
	Se os vetores $v_1, v_2, v_3, \dots, v_m$ geram o espaço vetorial $\textsf{E}$ e os vetores $u_1, u_2, u_3, \dots, u_n$ são LI, então $n \leq m$.
\end{theorem}

\begin{theorem}\bref{[Lima,AlgLin,2014]}
	Se o espaço vetorial $\textsf{E}$ admite uma base $\mathcal{B} = \{ u_1, u_2, u_3, \dots, u_n \}$ com $n$ elementos, qualquer outra base de $\textsf{E}$ possui também $n$ elementos.
\end{theorem}

\begin{definition}\bref{[Lima,AlgLin,2014]}
	Diz-se que o espaço vetorial $\textsf{E}$ tem dimensão finita quando admite uma base $\mathcal{B} = \{ v_1, v_2, v_3, \dots, v_n \}$ com um número finito $n$ de elementos. Este número, que é igual para todas as bases de $\textsf{E}$, chama-se a dimensão do espaço vetorial $\textsf{E}$. Denotado por $dim \textsf{E} := n$. Diz-se que o espaço vetorial $\textsf{E} = \{ 0_v \}$ tem dimensão zero
\end{definition}

\begin{theorem}\bref{[Lima,AlgLin,2014]}
	Se a dimensão de $\textsf{E}$ é $n$, um conjunto com $n$ vetores, gera $\textsf{E}$ se, e somente se, é LI.
\end{theorem}

\begin{theorem}\bref{[Lima,AlgLin,2014]}
	Seja $\textsf{E}$ um espaço vetorial de dimensão finita $n$, então:
	\begin{itemize}
		\item Todo conjunto $X$ de geradores de $\textsf{E}$ contém uma base.
		\item Todo conjunto LI $\{ v_1, v_2, v_3, \dots, v_n \} \subset \textsf{E}$ está contido numa base.
		\item Todo subespaço vetorial $\textsf{F} \subset \textsf{E}$ tem dimensão finita a qual $\leq n$.
		\item Se a dimensão do subespaço $\textsf{F} \subset \textsf{E}$ é igual a $n$, então $\textsf{F} = \textsf{E}$
	\end{itemize}
\end{theorem}

\begin{definition}\bref{[Lima,AlgLin,2014]}
	Diz-se que o espaço vetorial $\textsf{E}$ tem dimensão infinita quando não tem dimensão finita. Ou seja, quando nenhum subconjunto finito de $\textsf{E}$ é uma base.
\end{definition}

\begin{definition}\bref{[Lima,AlgLin,2014]}
	Diz-se que a variedade afim de $V \subset \textsf{E}$ tem dimensão $r$ quando $V = x + \textsf{F}$, onde o subespaço vetorial $\textsf{F} \subset \textsf{E}$.
\end{definition}


\section{Transformações Lineares}

\begin{definition}\bref{[Lima,AlgLin,2014]}
	Sejam $\textsf{E}, \textsf{F}$ espaços vetoriais. Uma transformação linear $A:\textsf{E} \rightarrow \textsf{F}$ é uma correspondência que associa a cada vetor $v \in \textsf{E}$ um vetor $A(v) = Av \in \textsf{F}$ de modo que valham, para quaisquer $u, v \in \textsf{E}$ e $\alpha \in \mathbb{R}$, as relações:
	\begin{itemize}
		\item $A(u + v) = Au + Av$
		\item $A(\alpha v) = \alpha  Av$
	\end{itemize}
\end{definition}

\begin{definition}\bref{[Lima,AlgLin,2014]}
	Seja $\mathcal{L}(\textsf{E}; \textsf{F})$, espaço vetorial dos conjunto das transformações lineares de $\textsf{E}$ para $\textsf{F}$. $\mathcal{L}(\textsf{E}) := \mathcal{L}(\textsf{E}; \textsf{E})$, $A: \textsf{E} \rightarrow \textsf{E}$ chamado de operador linear em $\textsf{E}$. E $\textsf{E}^* :=\mathcal{L}(\textsf{E}; \mathbb{R})$, $\varphi:\textsf{E} \rightarrow \mathbb{R}$, chamados funcionais lineares. O conjunto dos funcionais lineares $\textsf{E}^*$ de dual de $\textsf{E}$
\end{definition}

\begin{theorem}\bref{[Lima,AlgLin,2014]}
	Sejam $\textsf{E}$, $\textsf{F}$ espaços vetoriais e $\mathcal{B}$ uma base de $\textsf{E}$. A cada vetor $u \in \mathcal{B}$, façamos corresponder (de maneira arbitrária) um vetor $u' \in \textsf{F}$. Então existe uma única transformação linear $A:\textsf{E} \rightarrow \textsf{F}$ tal que $Au = u', \forall u \in \mathcal{B}$.
\end{theorem}

\begin{definition}\bref{[Lima,AlgLin,2014]}
	Dadas as transformações lineares $A:\textsf{E} \rightarrow \textsf{F}$, $B:\textsf{F} \rightarrow \textsf{G}$, $D(B) = CD(A)$. Define-se o produto $BA:\textsf{A} \rightarrow \textsf{G}, \forall v \in \textsf{E}, (\textsf{BA})v = \textsf{B}(\textsf{A}v)$
\end{definition}


Dadas as transformações lineares $A:\textsf{E} \rightarrow \textsf{F}$, $B:\textsf{F} \rightarrow \textsf{G}$, $C:\textsf{G} \rightarrow \textsf{H}$. Temos as seguintes propriedades:
\begin{description}
	\item[Associatividade] $(\textsf{C}\textsf{B})\textsf{A} = \textsf{C}(\textsf{B}\textsf{A})$
	\item[Distributiva à esquerda] $(\textsf{B} + \textsf{C})\textsf{A} = \textsf{B}\textsf{A} + \textsf{C}\textsf{A}$
	\item[Distributiva à direita] $\textsf{C}(\textsf{A} + \textsf{B}) = \textsf{C}\textsf{A} + \textsf{C}\textsf{B}$
	\item[Homogeneidade] $\textsf{B}(\alpha\textsf{A}) = \alpha(\textsf{B}\textsf{A})$
\end{description}

\begin{definition}\bref{[Lima,AlgLin,2014]}
	Um operador $A$ chama-se nilpotente quando, para algum $n \in \mathbb{N}$, tem-se $A^n = 0$. Um exemplo significativo de um operador nilpotente é a derivação $D: \textsf{P}_n \rightarrow \textsf{P}_n$. Para todo polinômio $p$ de grau $\leq n$ tem-se $D^{n + 1}p = 0$, logo $D^{n + 1} = 0$.
\end{definition}


\section{Núcleo e Imagem}

Considerando a transformação linear $A: \textsf{E} \rightarrow \textsf{F}$ sendo, $E, F$ dois espaços vetoriais.

\begin{definition}\bref{[Lima,AlgLin,2014]}
	A imagem de $A$ é o subconjunto $\mathcal{I}m(A)  := \{ w \in \textsf{F} : w = Av, \forall v \in \textsf{E}\}$.
\end{definition}


\begin{definition}\bref{[Lima,AlgLin,2014]}
	Se $\mathcal{I}m(A) = \textsf{F}$, diz que $A$ é sobrejetiva.
\end{definition}


\begin{definition}\bref{[Lima,AlgLin,2014]}
	Uma transformação linear $B: \textsf{F} \rightarrow \textsf{E}$ chama-se inversa à direita da transformação $A$, quando tem $AB = I_F$, ou seja, $A(Bw) = w, \forall w \in \textsf{F}$.
\end{definition}


\begin{theorem}\bref{[Lima,AlgLin,2014]}
	Uma transformação linear $A: \textsf{E} \rightarrow \textsf{F}$ entre espaços vetoriais de dimensões finitas possui uma inversa à direita se, e somente se $A$ é sobrejetiva.
\end{theorem}

\begin{definition}\bref{[Lima,AlgLin,2014]}
	O núcleo de $A$ é o subconjunto $\mathcal{N}(A)  := \{ v \in \textsf{E} : Av = 0\}$.
\end{definition}

\begin{definition}\bref{[Lima,AlgLin,2014]}
	Uma transformação linear $A: \textsf{E} \rightarrow \textsf{F}$ chama-se injetiva $\Leftrightarrow \forall v,v' \in \textsf{E}, v \neq v' \Rightarrow Av \neq Av', Av, Av' \in \textsf{F}$.
\end{definition}

\begin{theorem}\bref{[Lima,AlgLin,2014]}
	Uma transformação linear $A: \textsf{E} \rightarrow \textsf{F}$ é injetiva $\Leftrightarrow \mathcal{N}(A) = \{ 0_v \}$.
\end{theorem}

\begin{theorem}\bref{[Lima,AlgLin,2014]}
	Uma transformação linear $A: \textsf{E} \rightarrow \textsf{F}$ é injetiva $\Leftrightarrow$ leva vetores LI para vetores LI.
\end{theorem}

\begin{theorem}\bref{[Lima,AlgLin,2014]}
	Seja $A: \textsf{E} \rightarrow \textsf{F}$ uma transformação linear. $V = \{x \in \textsf{E} : Ax = b, \forall b \in \mathcal{I}m(A)\}$ é uma variedade afim paralela a $\mathcal{N}(A)$.
\end{theorem}


\begin{definition}\bref{[Lima,AlgLin,2014]}
	Uma transformação linear $B: \textsf{F} \rightarrow \textsf{E}$ chama-se inversa à esquerda da transformação $A$, quando tem $BA = I_E$, ou seja, $B(Aw) = w, \forall w \in \textsf{F}$.
\end{definition}

\begin{theorem}\bref{[Lima,AlgLin,2014]}
	Uma transformação linear $A: \textsf{E} \rightarrow \textsf{F}$ entre espaços vetoriais de dimensões finitas possui uma inversa à esquerda se, e somente se $A$ é injetiva.
\end{theorem}

\begin{definition}\bref{[Lima,AlgLin,2014]}
	Uma transformação linear $A: \textsf{E} \rightarrow \textsf{F}$ chama-se inversível quando existe $B: \textsf{F} \rightarrow \textsf{E}$ linear tal que $BA = I_E$ e $AB = I_F$, quando $B$ é ao mesmo tempo inversa pela esquerda e à direta de $A$. $B$ é inversa de $A$ e denota-se $B = A^{-1}$.
\end{definition}

\begin{theorem}\bref{[Lima,AlgLin,2014]}
	Se uma transformação linear $A$ é inversível é equivalente dizer:
	\begin{itemize}
		\item $A$ é injetiva e sobrejetiva ou seja, $A$ é uma bijeção linear entre $\textsf{E}$ e $\textsf{F}$.
		\item $A: \textsf{E} \rightarrow \textsf{F}$ é um isomorfismo e os espaços vetoriais $\textsf{E}$ e $\textsf{F}$ são isomorfos.
		\item $A$ tem uma inversa à esquerda $B:\textsf{F} \rightarrow \textsf{E}$ e uma inversa à direita $C:\textsf{F} \rightarrow \textsf{E}$ então $B = C$ e $A$ é um isomorfismo, com $A^{-1} = B = C$.
	\end{itemize}
\end{theorem}

\begin{theorem}\bref{[Lima,AlgLin,2014]}
	Uma transformação linear $A: \textsf{E} \rightarrow \textsf{F}$ e $B: \textsf{F} \rightarrow \textsf{G}$ temos:
	\begin{itemize}
	\item $(AB)^{-1} = B^{-1}A^{-1}$
	\item $(\alpha A)^{-1} = \frac{1}{\alpha}A^{-1}$, $\alpha \neq 0$
	\end{itemize}
\end{theorem}

\begin{theorem}\bref{[Lima,AlgLin,2014]}[Teorema do Núcleo e da Imagem]
	Sejam $\textsf{E}$, $\textsf{F}$ espaços vetoriais de dimensão finita. Para toda transformação linear $A:\textsf{E} \rightarrow \textsf{F}$ tem-se $dim\ \textsf{E} = dim\ \mathcal{N}(A) + dim\ \mathcal{I}m(A)$
\end{theorem}

\begin{corollary}\bref{[Lima,AlgLin,2014]}
	Sejam $\textsf{E}$, $\textsf{F}$ espaços vetoriais de \textbf{mesma dimensão finita}. Uma transformação linear $A:\textsf{E} \rightarrow \textsf{F}$ é injetiva se, e somente se, é sobrejetiva, portantdo é um isomorfismo.
\end{corollary}

\section{Produto Interno}

\begin{definition}\bref{[Lima,AlgLin,2014]}
	Um produto interno num espaço vetorial $\textsf{E}$ é um funcional bilinear simétrico e positivo em $\textsf{E}$. É uma função $\textsf{E}\times\textsf{E} \rightarrow \mathbb{R}$, que associa cada par de vetores $u, v \in \textsf{E}$ a um número real $\langle u, v\rangle$ de modo que sejam válidas as seguintes propriedades:
	\begin{description}
		\item [Bilinearidade] 
				\tabto{2cm} $\langle u + u', v\rangle = \langle u, v\rangle + \langle u', v\rangle$ \\
				\tabto{2cm} $\langle \alpha u, v\rangle = \alpha\langle u, v\rangle$ \\
				\tabto{2cm} $\langle u, v + v'\rangle = \langle u, v\rangle + \langle u, v'\rangle$ \\
				\tabto{2cm} $\langle u, \alpha v\rangle = \alpha\langle u, v\rangle$ \\
		\item [Comutatividade (simetria)]
				\tabto{2cm} $\langle u, v\rangle = \langle v, u\rangle$ 
		\item [Positividade]
				\tabto{2cm} $\langle u, u\rangle > 0, u \neq 0$ 
	\end{description}
	~\\
	\bref{[Coelho,AlgLin,2013]} Seja $\mathsf{V}$ um $\mathsf{K}$-espaço vetorial, onde $K = \mathbb{R}$ ou $K = \mathbb{C}$. Um produto interno sobre $\mathsf{V}$ é uma função $\langle, \rangle : \mathsf{V} \times \mathsf{V} \rightarrow \mathsf{K}$ que satisfaz as seguintes quatro propriedades:
	\begin{description}
		\item [P1] $\langle u + v, w\rangle = \langle u, w\rangle + \langle v, w\rangle, \forall u, v, w \in \mathsf{V}$ 
		\item [P2] $\langle \lambda u, v\rangle = \lambda\langle u, v\rangle, \forall \lambda \in \mathsf{K},  \forall u, v \in \mathsf{V}$ 
		\item [P3] $\langle u, v\rangle = \overline{\langle v, u\rangle}, \forall u, v \mathsf{V}$ 
		\item [P4] $\langle u, u\rangle > 0, u \in \mathsf{V}$ e $u \neq 0$ 
	\end{description}
\end{definition}
\bref{[Coelho,AlgLin,2013]} É possível deduzir:
	\begin{description}
		\item [] $\langle 0, v\rangle = \langle v, 0\rangle = 0, \forall v \in \mathsf{V}$ 
		\item [] $\langle v, v\rangle = 0 \Leftrightarrow v = 0$ 
		\item [P5] $\langle u, v + w\rangle = \langle u, v\rangle + \langle u, w\rangle, \forall u, v, w \in \mathsf{V}$
		\item [P6] $\langle u, \lambda v \rangle = \overline{\langle \lambda v, y\rangle} = \overline{\lambda} \overline{ \langle v, u\rangle} = \overline{\lambda} \langle u, v \rangle , \forall \lambda \in \mathsf{K},  \forall u, v \in \mathsf{V}$ 
	\end{description}
	
\begin{proposition}[Identidade de Polarização]\bref{[Coelho,AlgLin,2013]} 
	Seja $\mathsf{V}$ um $\mathsf{K}$-espaço vetorial com produto interno $\langle, \rangle$ e sejam $u, v \in \mathsf{V}$. \\
	Para $\mathsf{K} = \mathbb{R}$:
	\[
		\langle u, v \rangle = \frac{1}{4} \lVert u + v \rVert ^ 2 - \frac{1}{4} \lVert u - v \rVert ^ 2
	\] 
	Para $\mathsf{K} = \mathbb{C}$:
	\[
		\langle u, v \rangle = \frac{1}{4} \lVert u + v \rVert ^ 2 - \frac{1}{4} \lVert u - v \rVert ^ 2 +
		\frac{i}{4} \lVert u + iv \rVert ^ 2 - \frac{i}{4} \lVert u - iv \rVert ^ 2
	\] 
\end{proposition}

\begin{theorem}[Desigualdade de Cauchy-Schwarz]\bref{[Coelho,AlgLin,2013]} 
	Seja $\mathsf{V}$ um $\mathsf{K}$-espaço vetorial com produto interno $\langle, \rangle$. Então
\[
	|\langle u, v \rangle| \leq \lVert u \rVert \lVert v \rVert, \forall u, v\in \mathsf{V}.
\]
A igualdade $|\langle u, v \rangle| = \lVert u \rVert \lVert v \rVert$ é válida se e somente se $\{u, v\}$ for linearmente dependente.
\end{theorem}

\begin{corollary}[Desigualdade de Triangular]\bref{[Coelho,AlgLin,2013]} 
	Seja $\mathsf{V}$ um $\mathsf{K}$-espaço vetorial com produto interno $\langle, \rangle$. Então
\[
	\lVert u +  v \rVert \leq \lVert u \rVert  + \lVert v \rVert, \forall u, v\in \mathsf{V}.
\]
\end{corollary}

\begin{definition}\bref{[Lima,AlgLin,2014]}
	Seja $\textsf{E}$ um espaço vetorial com produto interno, $u, v \in \textsf{E}$. $u$ e $v$ são ortogonais $\Leftrightarrow$ $\langle u, v \rangle = 0$. Denotado como $u \perp v$.\\
	
	\bref{[Coelho,AlgLin,2013]} Seja $\mathsf{V}$ um $\mathsf{K}$-espaço vetorial com produto interno $\langle, \rangle$ e sejam $u, v \in \mathsf{V}$. Dizemos que $u$ e $v$ são ortogonais se $\langle u, v \rangle = 0$. Um subconjunto $\mathsf{A}$ de $\mathsf{V}$ é chamado de ortogonal se os seus elementos são ortogonais dois a dois e dizemos que $\mathsf{A}$ é um conjunto ortonormal se for um ortogonal e se $\lVert u \rVert = 1, \forall u \in \mathsf{A}$.
\end{definition}

\begin{definition}\bref{[Lima,AlgLin,2014]}
	Seja $\textsf{E}$ um espaço vetorial com produto interno. Um conjunto $X \subset \textsf{E}$. $\forall u, v \in X$ tal que $u \neq v \land u \perp v $ diz que $X$ é ortogonal. Se todos os vetores em $X$ são unitários $X$ é um conjunto ortonormal.
\end{definition}

\begin{theorem}\bref{[Lima,AlgLin,2014]}
	Num espaço vetorial $\textsf{E}$ com produto interno, todo conjunto ortogonal $X$ de vetores não nulos é LI.
\end{theorem}

\begin{proposition}\bref{[Coelho,AlgLin,2013]} 
	Seja $\mathsf{V}$ um $\mathsf{K}$-espaço vetorial com produto interno $\langle, \rangle$ e seja $\mathsf{A}$ um subconjunto ortogonal de $\mathsf{V}$ formado por vetores não nulos.
	\begin{enumerate}[label=(\alph*)]
		\item Se $v \in [ v_1, \dots, v_n ]$, com $v_i \in \mathsf{A}$, então:
		\[
			v = \sum_{i = 1}^{n} \frac{\langle v, v_i \rangle}{\lVert v_i \rVert ^ 2}v_i
		\]
		\item $\mathsf{A}$ é linearmente independente.
	\end{enumerate}
\end{proposition}

\begin{corollary}\bref{[Coelho,AlgLin,2013]} 
	Seja $\mathsf{V}$ um $\mathsf{K}$-espaço vetorial com produto interno $\langle, \rangle$ e seja $\{ v_1, \dots, v_n \}$ uma base ortonormal de $\mathsf{V}$. Então para $v \in \mathsf{V}$, temos
	\[
		v = \sum_{i = 1}^{n} \langle v, v_i \rangle v_i
	\]
\end{corollary}

\begin{theorem}\bref{[Coelho,AlgLin,2013]} 
	Todo espaço vetorial de dimensão finita $n \geq 1$ com produto interno possui uma base ortonormal.
\end{theorem}

\begin{corollary}\bref{[Coelho,AlgLin,2013]} 
	Seja $\mathsf{V}$ um $\mathsf{K}$-espaço vetorial munido de produto interno. Sejam $\mathsf{B} = \{ u_1,\dots, u_n \}$ e $\mathsf{B'} = \{ v_1,\dots, v_n \}$ duas bases ortonormais de $\mathsf{V}$. Se $M$ é a matriz de mudança de base $\mathsf{B}$ para $\mathsf{B'}$. então $M \overline{M ^ T} = \overline{M ^ T} M = Id_n$.
\end{corollary}

\begin{definition}\bref{[Coelho,AlgLin,2013]} 
	Seja $\textsf{V}$ um espaço vetorial com produto interno, e seja $\mathsf{S} \subseteq \mathsf{V}$ um subconjunto $\mathsf{V}$. Chamamos de ortogonal de $\mathsf{S}$ ao subconjunto $\mathsf{S} ^ \perp = \{ v \in \mathsf{V} | \langle v, u \rangle = 0, \forall u \in \mathsf{S} \}$.
\end{definition}

\begin{proposition}\bref{[Coelho,AlgLin,2013]} 
	Seja $\mathsf{V}$ um $\mathsf{K}$-espaço vetorial munido de produto interno. Sejam $\mathsf{W} \subseteq \mathsf{V}$ um subespaço e $\mathsf{B} = \{ w_1,\dots, w_k \}$ um gerador para $\mathsf{W}$. Então $v \in \mathsf{W} ^ \perp$ se e somente se $\langle v, w_i \rangle$, para cada $i = 1, \dots, k$.
\end{proposition}

\begin{proposition}\bref{[Coelho,AlgLin,2013]} 
	Seja $\mathsf{V}$ um $\mathsf{K}$-espaço vetorial de dimensão $n \geq 1$ e com produto interno e seja $\mathsf{W} \subsetneq \mathsf{V}$ um subespaço próprio de $\mathsf{V}$. Então $\mathsf{V} = \mathsf{W} \oplus \mathsf{W} ^ \perp$.
\end{proposition}

\begin{corollary}\bref{[Coelho,AlgLin,2013]} 
	Seja $\mathsf{V}$ um $\mathsf{K}$-espaço vetorial de dimensão finita com produto interno e seja $\mathsf{W} \subsetneq \mathsf{V}$ um subespaço de $\mathsf{V}$. Então 
	\[
		dim_K \mathsf{V} = dim_K \mathsf{W} + dim_K \mathsf{W} ^ \perp
	\]
\end{corollary}

\end{document}
